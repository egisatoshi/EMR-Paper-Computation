\textwidth 6in
\oddsidemargin.20in
\evensidemargin.35in
%\lineskip 1in
%\baselineskip.55cm

% Read in specially defined commands
% Change margins and baselinestretch in draft mode
\def\draft{
% make margins smaller
\addtolength{\oddsidemargin}{-0.5in}
\addtolength{\topmargin}{-0.65in}
\addtolength{\textheight}{1in}
\addtolength{\textwidth}{1.5in}
% A more reasonable baselinestretch, is still nearly doublespaced.
\def\baselinestretch{1.4}
}
%%%%%%%%%%%%%%%%%%%%%%%%%%%%%%%%%%%%%%%%%%%%%%%%%%%%%%%%
%%              Theorems                               %
%%%%%%%%%%%%%%%%%%%%%%%%%%%%%%%%%%%%%%%%%%%%%%%%%%%%%%%%
%\theoremstyle{definition}
%\theoremstyle{theorem}
\newtheorem{thm}{Theorem}[section]
\newtheorem{cor}[thm]{Corollary}
\newtheorem{lem}[thm]{Lemma}
\newtheorem{prop}[thm]{Proposition}
\theoremstyle{definition}
\newtheorem{defn}{Definition}[section]
\newtheorem{rem}{Remark}[section]
\newtheorem{conj}[thm]{Conjecture}
\newtheorem{exm}[thm]{Example}
\newtheorem{notation}{Notation}

%%%%%%%%%%%%%%%%%%%%%%%%%%%%%%%%%%%%%%%%%%%%%%%%%%%%%%%%
%%      Definitions and Commands                       %
%%%%%%%%%%%%%%%%%%%%%%%%%%%%%%%%%%%%%%%%%%%%%%%%%%%%%%%%
\newcommand{\A}{{\mathbb A}}
\newcommand{\Pdo}{{\Psi {\rm DO}}}
\newcommand{\LT}{{L(S^3 \times S^1)}}
\newcommand{\R}{{\mathbb R}}
\newcommand{\G}{{\rm Ell}_0^*}
\newcommand{\Q}{{\mathbb Q}}
\newcommand{\C}{{\mathbb C}}
\newcommand{\D}{{\mathbb D}}
\newcommand{\Z}{{\mathbb Z}}
\newcommand{\h}{{\mathbb H}}
\newcommand{\CP}{{\mathbb C}{\mathbb P}}
\newcommand{\I}{{\mathbb I}}
\newcommand{\N}{{\mathbb N}}
\newcommand{\Pf}{{\bf Proof: }}
\newcommand{\1}{{\bf 1}}
\newcommand{\acts}{{\circlearrowright}}
\newcommand{\g}{{\mathfrak  g}}
\newcommand{\uu}{{\mathfrak  u}}
\renewcommand{\h}{{\mathfrak  h}}
\newcommand{\ex}[1]{{e^{#1}}}
%%
\newcommand{\calA}{{\mathcal A}}
\newcommand{\calB}{{\mathcal B}}
\newcommand{\calC}{{\mathcal C}}
\newcommand{\calD}{{\mathcal D}}
\newcommand{\calE}{{\mathcal E}}
\newcommand{\calF}{{\mathcal F}}
\newcommand{\calG}{{\mathcal G}}
\newcommand{\calH}{{\mathcal H}}
\newcommand{\calI}{{\mathcal I}}
\newcommand{\calK}{{\mathcal K}}
\newcommand{\calM}{{\mathcal M}}
\newcommand{\calO}{{\mathcal O}}
\newcommand{\calP}{{\mathcal P}}
\newcommand{\calR}{{\mathcal R}}
\newcommand{\calS}{{\mathcal S}}
\newcommand{\calL}{{\mathcal L}}
\newcommand{\calX}{{\mathcal X}}
\newcommand{\calU}{{\mathcal U}}
\newcommand{\calV}{{\mathcal V}}
\newcommand{\calZ}{{\mathcal Z}}
%%
\renewcommand{\to}{\longrightarrow}
\newcommand{\dd}[1]{\frac{\partial}{\partial #1}}
\newcommand{\iso}{\stackrel{\simeq}{\longrightarrow}}
\renewcommand{\iff}{\text{if and only if }}
\newcommand{\spinc}{$\text{Spin}^c$}
\newcommand{\Dirac}{\not\!\!D}
\newcommand{\inc}{\hookrightarrow}
%%
\newcommand{\Ric}{\operatorname{Ric}}
\newcommand{\Aut}{\operatorname{Aut}}
\newcommand{\Hom}{\operatorname{Hom}}
\newcommand{\Hor}{\operatorname{Hor}}
\newcommand{\Ext}{\operatorname{Ext}}
\newcommand{\End}{\operatorname{End}}
\newcommand{\ev}{\operatorname{ev}}
\newcommand{\Map}{\operatorname{Map}}
\newcommand{\Diff}{\operatorname{Diff}}
\newcommand{\Tr}{\operatorname{Tr}}
\newcommand{\tr}{\operatorname{tr}}
\newcommand{\Lie}{\operatorname{Lie}}
\newcommand{\ad}{{\operatorname{ad\,}}}
\newcommand{\sign}{{\operatorname{sign}}}
\newcommand{\grad}{{\operatorname{grad}}}
\newcommand{\coker}{{\operatorname{coker}}}
\newcommand{\dett}{{\operatorname{det}}}
%\newcommand{\ch}{{\operatorname{ch}}}
\newcommand{\rk}{{\operatorname{rk}}}
\newcommand{\maxx}{{\operatorname{max}}}
\newcommand{\minn}{{\operatorname{min}}}
\newcommand{\id}{{\operatorname{id}}}
\newcommand{\ind}{{\operatorname{ind}}}
\newcommand{\Ind}{{\operatorname{Ind}}}
\newcommand{\spann}{{\operatorname{span}}}
\newcommand{\Spin}{{\operatorname{Spin}}}
\newcommand{\Pin}{{\operatorname{Pin}}}
\newcommand{\im}{{\operatorname{Im}}}
\renewcommand{\Im}{{\operatorname{Im}}}
\newcommand{\dimm}{{\operatorname{dim}}}
\newcommand{\cl}{{\operatorname{cl}}}
\newcommand{\CL}{{\mathcal{CL}}}%%
\newcommand{\tM}{{\tilde{M}}}
\newcommand{\lieg}{{\mathfrak{g}_\mathcal{G}}}
\newcommand{\tC}{{\tilde{C}}}
\newcommand{\tE}{{\tilde{E}}}
\newcommand{\tF}{{\tilde{F}}}
\newcommand{\talpha}{{\tilde{\alpha}}}
%%
\newcommand{\lap}{\Delta}
\newcommand{\weight}{(1+\Delta)^s}
\newcommand{\weightinv}{(1+\Delta)^{-s}}
\newcommand{\zbar}{\overline{z}}
\newcommand{\wbar}{\overline{w}}
\newcommand{\phibar}{\overline{\phi}}
\newcommand{\psibar}{\overline{\psi}}
\newcommand{\Bbar}{\overline{B}}
\newcommand{\Cbar}{\overline{C}}
\newcommand{\inv}{^{-1}}
\newcommand{\con}[2]{\nabla^{#1}_{#2}}
\newcommand{\chris}[4]{\Gamma^{#1,(#4)}_{#2#3}}
\newcommand{\trres}{P^{(-1)}}

%%      
\newcommand{\rb}[1]{\raisebox{0pt}[0pt][0pt]{#1}}

\newsavebox{\savepar}
\newenvironment{boxit}{\begin{lrbox}{\savepar}\begin{minipage}[b]{.5in}}
{\end{minipage}\end{lrbox}\fbox{\usebox{\savepar}}}

%%\newcommand{\ip}[1]{\langle\! #1 \! \rangle}
\newcommand{\ip}[1]{\langle #1 \rangle}
\newcommand{\norm}[1]{\| #1 \|}
\newcommand{\comment}[1]{\marginpar{\tiny{#1}}}
\newcommand{\printname}[1]{\smash{\makebox[0pt]{\hspace{-1.0in}\raisebox{8pt}{\tiny #1}}}}
\newcommand{\labell}[1] {\label{#1}\printname{#1}}
\newcommand{\bibitemm}[2]{\bibitem[#1]{#2}\mf{#2}}
\newcommand{\Name}[1]  {\smallskip\noindent{\bf #1 \ }}
\numberwithin{equation}{section}
\renewcommand{\theequation}{{\thesection{.}}\arabic{equation}}
\newcounter{dummy}
%%%%%%%%%%%%%%%%%%%%%%%%%%%%%%%%%%%%%%%%%%%%%%%%%%%%%
%  Printing of label keys
%
%  Label keys are printed (in a box) in the margin when the command
%           \labelon (below)
%  is activated.
%  This feature makes it easier to find labels for \ref commands
%  when working on draft.
%
\newcounter{labelflag} \setcounter{labelflag}{0}
\newcommand{\labelon}{\setcounter{labelflag}{1}}
\newcommand{\Label}[1]{
                       \ifnum\thelabelflag=1
                          \ifmmode
                             \makebox[0in][l]{\qquad\fbox{\rm#1}}
                          \else
                             \marginpar{\vspace{0.7\baselineskip}
                                        \hspace{-1.1\textwidth}
                                        \fbox{\rm#1}}
                          \fi
                       \fi
                       \label{#1}
                      }
\labelon
%%%%%%%%%%%%%%%%%%%%%%%%%%%%%%%%%%%%%%%%%%%%%%%%%%%%%%





